
\documentclass[a4paper,journal]{IEEEtran}

\usepackage{cite}
% \usepackage[nocompress]{cite}
\usepackage{ifpdf}

\ifpdf
\usepackage[pdftex]{graphicx}
\graphicspath{{./img/}}
\DeclareGraphicsExtensions{.pdf}
\else
\usepackage[dvips]{graphicx}
\graphicspath{{./img/}}
\DeclareGraphicsExtensions{.eps}
\fi

\usepackage[cmex10]{amsmath}
\usepackage{amsfonts}
\usepackage{amssymb}
\interdisplaylinepenalty=2500

\usepackage[margin=15mm]{geometry}

\usepackage{algorithmic}

\usepackage{array}

\usepackage{mdwmath}
\usepackage{mdwtab}

\usepackage{eqparbox}

\usepackage[hang,small,center,bf]{caption}
% \usepackage[tight,normalsize,sf,SF]{subfigure}
%\usepackage[tight,footnotesize]{subfigure}
\usepackage{subfig}
% \usepackage[caption=false,font=normalsize,labelfont=sf,textfont=sf]{subfig}
% \usepackage[caption=false,font=footnotesize]{subfig}

\usepackage[utf8x]{inputenc}
\usepackage[czech]{babel}
\usepackage[colorlinks=true,urlcolor=blue]{hyperref}
\usepackage{url}
\usepackage{fixltx2e}
\usepackage{stfloats}
\usepackage{ucs}
\usepackage{multirow}

% correct bad hyphenation here
\hyphenation{op-tical net-works semi-conduc-tor}

\renewcommand{\labelitemi}{$\bullet$}
\renewcommand{\labelitemii}{$\circ$}
\renewcommand{\labelitemiii}{$\ast$}

\setlength{\textheight}{260mm}

\begin{document}

\title{Návrh a verifikace řídícího systému pro nádraží}
\date{August 8, 2012}
\author{Peter~Boráros %
\thanks{{Peter Boráros}, CTU FEE,
see~\url{http://www.pborky.sk/contact} for a contact infomation}}%

% The paper headers
%\markboth{Peter Boráros, Czech technical university, Faculty of Electrical Engineering, Prague, Czech Republic}{}

\IEEEcompsoctitleabstractindextext{%
\begin{abstract}
Tento dokument popisuje řešení semestrální úlohy pro kurz a4m33au - automatické uvažování.
Cílem je návrh a verifikace řídícího systému pro vlakové nádraží, s použitím automatického
dokazování v logice prvního řádu.
\end{abstract}}

\maketitle
\IEEEdisplaynotcompsoctitleabstractindextext
\IEEEpeerreviewmaketitle

\section{Úvod}\label{sec:intro}
Táto práce představuje návrh a verifikaci řídícího systému pro vlakové nádraží, s použitím automatického
dokazování v logice prvního řádu.

V sekci \ref{sec:intro} je popsané podrobné zadání problému (je převzaté z webových stránek kurzu). 
V sekci \ref{sec:formal} přecházi přes jednotlivé aspekty problému, formalizuje je a přináší obecné řešení. 
Sekce \ref{sec:imple} popisuje implementaci a způsob použití navrženého systému. V sekci 
\ref{sec:exp} je ukážka experimentů, nad jednoduchými testovacími daty.

\subsection{Specifikace problému}
Nádraží je souvislý orientovaný graf. Uzly, z kterých nevedou šipky, nazveme výjezdy, uzly, 
do kterých nevedou vedou šipky, nazveme vjezdy. Omezujeme se na grafy, u kterých 
z každého vjezdu existuje cesta do každého výjezdu. Každý uzel a každá hrana mají 
zadaný unikátní název (začínající malým písmenem).

\subsubsection{Vlastnosti nádraží}
\begin{itemize}
	\item Každý uzel s více než jednou výstupní hranou je zároveň výhybka.
	\item Časově variabilní prvky v nádraží jsou:
	\begin{itemize}
		\item pohybující se vlaky,
		\item na vstupních uzlech řízená návěstidla a
		\item výhybky.
	\end{itemize}
	\item V daném časovém okamžiku je každý vlak právě v jednom uzlu.
	\item Vlaky se pohybují pouze ve směru orientace hran grafu (tj. nemohou couvat).
	\item Je-li v uzlu vlak, platí, že vlak někdy do uzlu přijel (nebyl tam od nepaměti) 
	a jednou odjede (ale není určeno, kdy - je to na ,,rozhodnutí strojvedoucího``).
	\item Na vjezdových uzlech (a pouze tam) jsou návěstidla. Ta blokují odjezd vlaků 
	ze vstupních uzlů: Je-li návěstidlo zavřené, vlak zůstává na vstupním uzlu; 
	je-li otevřené, může (ale nemusí) vyjet. Vlak (strojvedoucí) vždy tato návěstidla 
	respektuje. Každý vjezdový uzel má právě jednu výstupní hranu, není tedy nikdy výhybkou.
	\item Každý vlak má dán výstupní uzel, do kterého chce dojet. Tento cíl se nemění 
	celou dobu, co vlak projíždí nádražím.
	\item Nádraží podle tohoto cíle směruje vlak pomocí přepínání výhybek. 
	Řídící systém nádraží může libovolně nastavovat stav výhybek.
	\item Pokud je vjezdový uzel prázdný, může se v něm kdykoliv objevit nový 
	přijíždějící vlak (i hned po odjezdu předcházejícího).
\end{itemize}

\subsubsection{Kritické stavy v nádraží}
V nádraží rozlišujeme tyto kritické stavy:
\begin{itemize}
	\item Vlak stojí v uzlu (který je zároveň výhybkou, viz výše), a dojde k přepnutí výhybky.
	\item Dva nebo více vlaků přijede do stejného uzlu.
	\item Vstupní návěstidlo zůstane trvale uzavřené.
\end{itemize}

\subsubsection{Vstup}
Graf nádraží bude zadáván v následujícím formátu 
(podmnožina jazyka DOT \cite{Graphviz}):

\begin{verbatim}
digraph nadrazi_1 { 
  vjezd1 -> uzel1; 
  ...
  uzel1 -> vyjezd1; 
  uzel1 -> vyjezd2; 
} 
\end{verbatim}

\subsubsection{Výstup}
Úkolem je navrhnout program, který
\begin{itemize}
	\item pro zadané nádraží navrhne řídící systém a zformalizuje podle uvedeného zadání;
	\item dokáže, že navržený řídící systém pracuje správně, tj. že se nádraží nemůže dostat 
	do kritického stavu.
\end{itemize}

\subsubsection{Automatické dokazování}
Nádraží je modelováno v diskrétním čase. Čas je lineární, a každý časový okamžik má právě jeden následující a jeden předcházející. V každý časový okamžik si řídící systém nádraží určuje stavy výhybek a návěstidel.

Úkoly, které postupně zpracuje program pro libovolné nádraží pomocí nástrojů pro automatické dokazování:

\begin{itemize}
	\item Formalizace nádraží:
	\begin{itemize}
		\item Zformalizovat v logice 1. řádu v jazyce TPTP ,,fyzikální chování`` nádraží, 
		tedy jak vlaky projíždějí nádražím na základě návěstidel a ,,rozhodování strojvedoucích``. 
		Každý predikát p závislý na čase popisující něco, co dovedeme určit, musí být popsaný nejvýše 
		jednou formulí tvaru $p(T+1) \Leftrightarrow \phi$, kde $\phi$ je formule závislá pouze 
		na okolnostech v čase $T$ a dřívějších (tím je syntakticky zaručena korektnost definice).
		Do toho spadá zejména stav výhybek, zda je v daném uzlu vlak, atd. 
		Nespadá sem především vůle strojvedoucího, kterou neznáme
		(jen víme, že vždy nakonec s vlakem odjede).
		\item Ukázat, že tato formalizace není sporná s přidanými podmínkami, že strojvůdce vlaku jede hned,
		jakmile může, a že do nádraží vjede vlak vždy, jakmile může.
	\end{itemize}
	\item Zformalizovat návrh řídícího systému stejným způsobem jako v predešlém bodě.
	\item Ukázat, že je výsledná formalizace nádraží a jeho řízení bezesporná.
	\item Dokázat, že nikdy nenastane kritický stav.
	\item Nádraží musí pouštět vlaky hned, jakmile je to možné. 
	Je třeba dokázat pro toto nádraží, že budou-li v čase t v tomto nádraží 2 vlaky, 
	jeden vlak na out1 a jeden na in, že se návěstidlo v in v čase t+1 otevře.
\end{itemize}


\section{Formalizace problému}\label{sec:formal}
\subsection{Reprezentace grafu v logice prvního řádu}
Vlakové nádraží je popsané orientovaným grafem. V níže popsané logické struktuře vrcholy grafu
představují konstantné symboly (napr. vrchol $a$ je popsaný symbolem $a/0$).
Orientované hrany jsou popsane binárním predikátem $edge/2$. Term $edge(a,b)$ říká, že v grafu je přítomna 
hrana $\langle a,b\rangle$ a naopak neprítomnost této hrany je určena termem $\neg edge(a,b)$.
Pro úplný popis grafu je potrebné specifikovat, které hrany jsou přítomny, a které přítomny nejsou,
t.j.:
\begin{equation}
\left( \bigwedge_{\langle a,b\rangle\in G}{edge(a,b)} \right) \wedge 
\left( \bigwedge_{\langle a,b\rangle\not\in G}{\neg edge(a,b)} \right)\;.
\end{equation}

Dále nasleduje definice orientované cesty v grafu, reprezentované predikátem $path/2$. Hrana je zároven (elementární) cesta:
\begin{equation}
\forall a,b: edge(a,b) \Rightarrow path(a,b)\;,
\end{equation}
a dále, cesta je tranzitivní:
\begin{equation}
	\forall a,b,c: path(a,b)\wedge path(b,c) \Rightarrow path(a,c)\;.
	%\footnote[1]{z důvodu lepší citelnosti jsou opomenuty závorky a budeme spoléhat na pravidla asociativity }
\end{equation}


Jestě je potřeba popsát vstupní, výstupní a divergetní uzly.
Predikát $input/1$ je definován:
\begin{equation}
\forall x: input(x) \Leftrightarrow \bigvee_{y \in in(G)} (x=y)\;,
\end{equation}
kde funkce $in$ vrací množinu uzlů, do kterých nevede žádná hrana. Dále predikát $output/1$:
\begin{equation}
\forall x: output(x) \Leftrightarrow \bigvee_{y \in out(G)} (x=y)\;,
\end{equation}
kde funkce $out$ vrací množinu uzlů, ze kterých nevede žádná hrana. A koněčne predikát $diverge/1$:
\begin{equation}
\forall x: diverge(x) \Leftrightarrow \bigvee_{y \in more(G)} (x=y)\;,
\end{equation}
kde funkce $more$ vrací množinu uzlů, s více než jedním potomkem.


\subsection{Definice diskrétního času}
Nejprve je potřebné definovat predikát lineárního uspořádaní $less$. 
Ten je definovaný následujícímy axiomy:
\begin{equation}\label{eq:ltlantisym}
\forall x,y:  less\left(x,y\right) \wedge less\left(y,x\right)\Rightarrow \left(x = y\right)
\end{equation}
\begin{equation}\label{eq:ltltran}
\forall x,y,z: less\left(x,y\right) \wedge less\left(y,z\right) \Rightarrow less\left(x,z\right) 
\end{equation}
\begin{equation}\label{eq:ltltotal}
\forall x,y: less\left(x,y\right) \vee less\left(y,x\right)
\end{equation}
Vztahy (\ref{eq:ltlantisym}), (\ref{eq:ltltran}) a (\ref{eq:ltltotal}) představují antisymetrii,
tranzitivitu a úplnost lineárního uspořádání.

S pomocí výše uvedeného predikátu $less/2$ definujeme funkci $succ/1$, která představuje přímeho následníka:
\begin{equation}
\begin{split}
\forall x:less\left(x,succ\left(x\right)\right) \wedge
\left(\forall y:less\left(y,x\right) \vee 
less\left(succ\left(x\right),y\right)\right)
\end{split}
\end{equation}
a dále musí platit:
\begin{equation}
\forall x: succ(x) \not = x
\end{equation}

Funkci $succ$ je možné použít k vyjádření následujícího časového okamžiku.
Napr. hodnota $succ(succ(T))$ představuje posunutí o dva okamžiky vpřed proti hodnote $T$.

\subsection{Pohyb vlaku}
Poloha a smeřování vlaku v čase jsou určeny predikátem $at/3$; term $at(t,y,u)$ říká, že v čase $t$, v uzlu $y$ je vlak směřující do
uzlu $u$. Víme, že vlak směřující do uzlu $u$ je v nejakém uzlu $y$ v nejakém čase $succ(t)$ tehdy a jen tehdy, byl-li v daném uzlu
v předešlém okamžiku a ,,nechtěl`` nebo nemohl z neho vyjet, anebo byl v předešlém uzlu a mohl a zároveň ,,chtěl`` vyjet:
\begin{equation}\label{eq:move}
\begin{split}
\forall t,y,u:&\; at\left(succ(t),y,u\right) \Leftrightarrow\\
&(at(t,y,u) \wedge \neg (want(t,y) \wedge \exists z:may(t,y,z))) \vee\\
&(at(t,x,u) \wedge want(t,x) \wedge may(t,x,y))\\ 
\end{split}
\end{equation}
Predikát $want/2$ představuje ,,vůli`` strojvůdce, t.j. $want(t,x)$ znamená, že v čase $t$ se ,,chce`` posunout z uzlu $x$ dále.
Predikát $may/3$ představuje možnost pokračovat, t.j. je-li u vstupů otevřené návěstidlo, případne u divergentního spojení sepnuta výhybka v daném směru, u ostatních uzlú možnosť pokračovat není omezena - 
vlak může vyjet jakmile strojvůdce ,,chce``.
Term $may(t,x,y)$ znamená, že vlak může v čase $t$ postoupit z uzlu $x$, do uzlu $y$. 
\begin{equation}
\begin{split}
\forall t,y,u: may(t,x,y) \Leftrightarrow &\; edge(x,y) \wedge \\
( (input(x) \wedge &\; signal(t,x) ) \vee \\
(diverge(x) \wedge &\; branch(t,x,y)) \vee \\
(\neg diverge(x) \wedge &\; \neg input(t,x,y) ))
\end{split}
\end{equation}

Dle zadání víme, že vlak do uzlu jednou přišel a taky, že jednou odjede. ,,Vůle`` strojvůdce je tedy možno definovat
vzhledem k výskytu vlaku v nejakém uzlu:
\begin{equation}
\begin{split}
&\forall t,x: ((\exists u: at(t,x,u)) \Rightarrow \\ 
&\Rightarrow\exists t_1: less(t,t_1) \wedge (\forall t_2: less(t_1,t_2) \wedge want(t_2,x)))\;,\\
\end{split}
\end{equation}
t.j. je-li vlak v uzlu $x$ a v čase $t$ tak existuje okamžik v budoucnosti, od kterého platí, že
strojevůdce ,,chce`` odjet.

Je ale taky nutné ověřit jestli je formalizace není sporná, když vlak vyjede jakmile to je možné. V našem
případe to znamená:
\begin{equation}
\forall t,x: (\exists u: at(t,x,u)) \Rightarrow want(t,x)\;.
\end{equation}

\subsection{Kritické stavy}
Kritický stav v nejakem čase $t$ je určen termem $crit(t)$. Ke kritickému stavu může dojít v 
následujích případech:
\subsubsection{Návěstidlo zůstane trvale uzavřené}
\begin{equation}
\begin{split}
\forall t: (\exists x,u: input(x) \wedge at(t,x,u)& \;\wedge\\
 (\neg \exists t_1: less(t,t_1) \wedge signal(t_1,x))) &\Rightarrow crit(t)
\end{split}
\end{equation}
\subsubsection{Výhybka se přepne v okamžiku, kdy v uzlu je vlak}
\begin{equation}
\begin{split}
\forall t: (\exists x,u: diverge(x) \wedge at(t,x,u)& \;\wedge\\
 (\neg \exists y,z: y\not=z \wedge branch(t,x,y) & \;\wedge  \\
 \wedge branch(succ(t),x,z) )) &\;\Rightarrow crit(succ(t))
\end{split}
\end{equation}
\subsubsection{Do uzlu vjede víc než jeden vlak}
\begin{equation}
\begin{split}
&\forall t: (\exists y,u: \\
&\;(at(t,y,u) \wedge \neg (want(t,y) \wedge (\exists z: may(t,y,z)))\wedge \\
\wedge&\;(\exists x:  at(t,x,u) \wedge want(t,x) \wedge may(t,x,y)))) \Rightarrow\\
 \Rightarrow &\;crit(succ(t))\\
\end{split}
\end{equation}

\subsection{Řízení}
Řídící systém řídí nadraží signalizací na vstupu - predikát $signal/2$, 
a překlápěním výhybky - predikát $branch/3$. Vlak smí vyjet ze vstupu $x$ a v čase $t$ 
pouze tehdy, platí-li $signal(t,x)$.
Platí-li $branch(t,x,y)$ vlak může projít z uzlu $x$ do uzlu $y$ (za předpokladu, že jsou spojené hranou).

U řízení vstupních signálů je potřebné prihlédnout také k možnému kritickému stavu, kdy některý vstup
zůstane trvale uzavřen. Tomu je možné predejít použitím časovače (predikát $flop/2$). 
Povolení k výjezdu na vstupu $x$ může nastat pouze v případe platnosti $flop(t,x)$.
Pokud má nádraží dva vstupy - $X$ a $Y$ , tak platí:
\begin{equation}
\forall t: (flop(t,X)\wedge\neg flop(t,Y))\vee (\neg flop(t,X)\wedge flop(t,Y))\;,
\end{equation}
t.j. nanejvýš jeden vstup může být aktivní.
Pro dva stejné uzly dále platí:
\begin{equation}
\begin{split}
&\forall t: flop(t,X) \Rightarrow \\
&\Rightarrow(\neg flop(succ(t),X) \wedge flop(succ(t),Y))
\end{split}
\end{equation}
a
\begin{equation}
\begin{split}
&\forall t: flop(t,Y) \Rightarrow \\
&\Rightarrow(\neg flop(succ(t),Y) \wedge flop(succ(t),X))\;,
\end{split}
\end{equation}
t.j. dochází k ,,přepínání`` vstupů.

Dále je potřeba vyloučit možnost kolize vlaků. 
Predikát $block(t,x)$ říká, že daném čase $t$ je v nádraží vlak v takovém místě, 
které je dosažitelné z daného vstupu $x$.
 V případě, že by byl vypuštěn další vlak, mohlo by dojít ke kolizi.
\begin{equation}
\begin{split}
\forall t,z:& (input(z) \wedge (\exists x,u: at(t,x,u) \wedge \neg input(t,x) \wedge\\
\wedge&\neg (\exists y: path(x,y) \wedge path(y,z) ) ) \Rightarrow block(t,z)\\
\end{split}
\end{equation}

Koněčně můžeme napsat axiom řízení signalizace:
\begin{equation}
\forall t,x: (input(x) \wedge flop(t,x) \wedge \neg block(t,x)) \Rightarrow signal(t,x)
\end{equation}

\section{Implementace}\label{sec:imple}
V této sekci následuje popis prostředí pro automatizovanou vefifikaci popsaného problému.

Vstupem pro tento systém je jednoduchý popis nádraží ve formátu DOT \cite{Graphviz} a výstupem 
jsou logické formule ve formátu TPTP\cite{TPTP}.

Systém byl vytvořen v jazyku Python 2.7.3 \cite{Python}. Dále je použitý nástroj GNU Make 3.81 \cite{GNUMake}.

Pro účely této práce byla vytvořena knihovna \textbf{pytptp} pro manipulaci s derivačním stromem logických formulí.
Formule a její elementy jsou reprezentovány objekty. Je využito přetížení aritmetických a logických operací aby byl zápis
formulí čitelnejší. Formule může být převedena do formátu TPTP anebo jiných (napr. TEX).

\section{Experimenty} \label{sec:exp}



\bibliographystyle{plain}	% (uses file "plain.bst")
\bibliography{references}

\onecolumn
\clearpage
\appendix


\section{}
\subsection{}
\subsubsection{Axiomy LTL}
\paragraph*{}{Antisymetrie lineárního uspořádání.}
\begin{equation}
\begin{split}
\forall X,Y:&\left( \left( less \left(X,Y\right) \wedge less\left(Y,X\right) \right) \Rightarrow \left(X = Y\right) \right)
\end{split}
\end{equation}
\paragraph*{}{Tranzitivita lineárního uspořádání.}
\begin{equation}
\begin{split}
\forall X,Y,Z:&\left( \left( less\left(X,Y\right) \wedge less\left(Y,Z\right) \right) \Rightarrow less\left(X,Z\right) \right)
\end{split}
\end{equation}
\paragraph*{}{Úplnost lineárního uspořádání.}
\begin{equation}
\begin{split}
\forall X,Y:&\left(less\left(X,Y\right) \vee less\left(Y,X\right)\right)
\end{split}
\end{equation}
\paragraph*{}{Definice relace přímého následníka.}
\begin{equation}
\begin{split}
\forall X:&\left( less\left(X,succ\left(X\right)\right) \wedge \left(\forall Y:\left(less\left(Y,X\right) \vee less\left(succ\left(X\right),Y\right)\right)\right)  \right)
\end{split}
\end{equation}
\paragraph*{}{Definice relace přímého následníka.}
\begin{equation}
\begin{split}
\forall X:&\left(succ\left(X\right) \neq X\right)
\end{split}
\end{equation}


\subsubsection{Axiomy grafu}
\paragraph*{}{Definice hran které jsou přítomné a nepřítomné v grafu.}
\begin{equation}
\begin{split} &
\left(\bigwedge_{\langle T,H\rangle \in G}{H,T}: edge\left(T,H\right)\right) \wedge 
\left(\bigwedge_{\langle T,H\rangle \not \in G}{H,T}: \neg edge\left(T,H\right)\right)
\end{split}
\end{equation}
\paragraph*{}{Elementární cesta v grafu.}
\begin{equation}
\begin{split}
\forall X,Y:&\left( edge\left(X,Y\right) \Rightarrow path\left(X,Y\right) \right)
\end{split}
\end{equation}
\paragraph*{}{Tranzitivita cesty.}
\begin{equation}
\begin{split}
\forall X,Y,Z:&\left(\left( path\left(X,Z\right) \wedge path\left(Z,Y\right) \right) \Rightarrow path\left(X,Y\right) \right)
\end{split}
\end{equation}
\paragraph*{}{Enumerace vstupních uzlů.}
\begin{equation}
\begin{split}
\forall X:&\left(input\left(X\right) \Leftrightarrow \left(\bigvee_{Y \in roots\left(G\right) }{\left(X=Y\right)} \right)\right)
\end{split}
\end{equation}
\paragraph*{}{Enumerace divergentních spojení.}
\begin{equation}
\begin{split}
\forall X:&\left(diverge\left(X\right) \Leftrightarrow \left(\bigvee_{
Y \in G \wedge\left| children(Y) \right| > 1 
}{(X=Y)} \right)\right)
%\forall X:\left((diverge\left(X\right) \Leftrightarrow ((X = b) \vee (X = d)))\right)
\end{split}
\end{equation}


\subsubsection{Axiomy pohybu}
\paragraph*{}{Definice polohy vlaku v závislosti na předešlé poloze, vůli rušňovodiče a stavu semaforů.}
\begin{equation}
\begin{array}{rl}
\forall T,Y,U: 
	(at\left(succ\left(T\right),Y,U\right) \Leftrightarrow 
		& \left( at\left(T,Y,U\right) \wedge \left( 
			\neg want\left(T,Y\right) \vee 
			\left( 
				input\left(Y\right) \wedge \neg signal\left(T,Y\right)
			\right) 
		\right)
	\right) \vee \\
	& \qquad\vee\;(\exists X: ( 
		at\left(T,X,U\right) \wedge edge\left(X,Y\right) \wedge want\left(T,X\right) \wedge \\ 
		&  \qquad \qquad\qquad \wedge \; (\left(
				input\left(X\right) \wedge signal\left(T,X\right)
			\right) \vee \\
		&  \qquad \qquad\qquad \quad\vee \left(
				diverge\left(X\right) \wedge branch\left(T,X,Y\right)
			\right) \vee \\
		& \qquad \qquad\qquad \quad\vee \left(
				\neg input\left(X\right) \wedge \neg diverge\left(X\right)
			\right) \;))) \;) \\
\end{array}
\end{equation}


\subsubsection{Axiomy kritického stavu}
\paragraph*{}{Definice srážky vlaků. Vlaky se srazí, když vlak vjede na kolej, na které je jiný vlak, ktorý nemůže odjet.}
\begin{equation}
\begin{split}
\forall T:
(
	crit\left(succ\left(T\right)\right) \Leftarrow &\;
	\exists X,Y:(
		at\left(T,X,U\right) \wedge 
		( \neg want\left(T,X\right) \vee 
			( 
				input\left(Y\right) \wedge \neg signal\left(T,Y\right) 
			)
		) \wedge\\
		&\qquad\qquad\;\wedge \exists Y,V: (  at\left(T,Y,V\right) \wedge edge\left(Y,X\right) \wedge want\left(T,Y\right) \wedge \\
		&	\qquad\qquad\;\wedge (
				( input\left(Y\right) \wedge signal\left(T,Y\right)  ) \vee \\
		&		\qquad\qquad\qquad\;\vee ( diverge\left(Y\right) \wedge branch\left(T,Y,X\right) ) \vee \\
		&		\qquad\qquad\qquad\;\vee ( \neg input\left(Y\right) \wedge \neg diverge\left(Y\right)  )
			)
		)
	)
)\\
\end{split}
\end{equation}



\begin{equation}
\begin{split}
\forall T:\left((\exists X,U:\left((diverge\left(X\right) \wedge at\left(T,X,U\right) \wedge \exists Y,Z:\left(((Z \neq Y) \wedge branch\left(T,X,Y\right) \wedge branch\left(succ\left(T\right),X,Z\right))\right))\right) \Rightarrow crit\left(succ\left(T\right)\right))\right)
\end{split}
\end{equation}


\begin{equation}
\begin{split}
(\exists T,X,U:\left((input\left(X\right) \wedge at\left(T,X,U\right) \wedge \neg \exists T1:\left((less\left(T,T1\right) \wedge signal\left(T1,X\right))\right))\right) \Rightarrow crit\left(T\right))
\end{split}
\end{equation}

\subsubsection{Axiomy řízení}
\paragraph*{}{Blokace vstupního uzlu vlakem, který není ve vstupním uzlu.}
\begin{equation}
\begin{split}
\forall T,Z:
(
block\left(T,Z\right)  \Leftarrow &\;
(
	input\left(Z\right) \wedge \\
	&\wedge \exists X:
		( \neg input\left(X\right) \wedge (Z \neq X) 
			\wedge \exists U: \left( at\left(T,X,U\right) \right) \wedge \\
			& \qquad\qquad\wedge \neg \exists Y: \left( path\left(X,Y\right) \wedge path\left(Y,Z\right) \right) 
		)
)
)
\end{split}
\end{equation}
\paragraph*{}{Singalizace na 1. vstupu zněmožní výjezd vlaku, který je blokovaný.}
\begin{equation}
\begin{split}
\forall T:\left( signal\left(T,a1\right) \Leftarrow ( \neg block\left(T,a1\right) \wedge \exists U:\left( at\left(T,a1,U\right) \right) ) \right)
\end{split}
\end{equation}
\paragraph*{}{Singalizace na 2. vstupu zněmožní výjezd vlaku, který je blokovaný a když je na 1. vstpu čekající vlak.}
\begin{equation}
\begin{split}
\forall T:\left( signal\left(T,a2\right) \Leftarrow ( \neg block\left(T,a2\right) \wedge \neg \exists U:\left(at\left(T,a1,U\right)\right) \wedge \exists U:\left(at\left(T,a2,U\right)\right) ) \right)
\end{split}
\end{equation}
{(Obdobně pro 3. vstup a další vstupy.)}
\paragraph*{}{Výhybka se nastaví do polohy takové, aby umožnila průjezd vlaku do cíle.}
\begin{equation}
\begin{split}
\forall T,Z,Y:
(
branch\left(T,Z,Y\right) \Leftarrow &\;
(
diverge\left(Z\right) \wedge 
edge\left(Z,Y\right) \wedge \\
&\wedge \exists X,U:
( at\left(T,X,U\right) \wedge \\
&\qquad\qquad\wedge ( path\left(X,Z\right) \vee (X = Z) ) \wedge \\
&\qquad\qquad\wedge ( path\left(Y,U\right) \vee (Y = U) )
)
)
)\\
\end{split}
\end{equation}
\paragraph*{}{Vlak vyjede jakmile je to možné.}
\begin{equation}
\begin{split}
\forall T,X,U:\left((at\left(T,X,U\right) \Rightarrow want\left(T,X\right))\right)
\end{split}
\end{equation}

\subsubsection{Domněnky}
\paragraph*{}Je-li vlak na 1. vstupu a není blokovaný jeho výjezd, pak tento vlak někdy v budoucnosti dojede do cíle.
\begin{equation}
\begin{split}
\forall T:\left(((at\left(T,a1,e2\right) \wedge \neg block\left(T,a1\right)) \Rightarrow \exists T1:\left((less\left(T,T1\right) \wedge (T \neq T1) \wedge \exists U:\left(at\left(T1,e2,U\right)\right))\right))\right)
\end{split}
\end{equation}
\paragraph*{}Je-li vlak na 2. vstupu a není blokovaný jeho výjezd, pak tento vlak někdy v budoucnosti dojede do cíle.
\begin{equation}
\begin{split}
\forall T:(((at\left(T,a2,e2\right) \wedge \neg block\left(T,a2\right) \wedge &\neg \exists U:\left(at\left(T,a1,U\right)\right)) \Rightarrow \\
&\Rightarrow\exists T1:\left(less\left(T,T1\right) \wedge (T \neq T1) \wedge \exists U:\left(at\left(T1,e2,U\right)\right))\right))\\
\end{split}
\end{equation}
\paragraph*{}Jsou-li vlaky na vstupu neustále, tak v budoucnu nedojde ke kolizi.
\begin{equation}
\begin{split}
\forall T:\left(\exists X,U:\left((at\left(T,X,U\right) \wedge input\left(X\right)\right) \Rightarrow 
\neg\exists T1:\left(less\left(T,T1\right) \wedge crit\left(T1\right))\right)\right)
\end{split}
\end{equation}


% that's all folks
\end{document}
