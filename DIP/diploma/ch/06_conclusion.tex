

\chapter{Assessment and Conclusion}\label{ch:conclude}
%\section{} 
\section{Results}
\label{sec:results}

Analysis has been performed using non-public \code{NetFlow} and \code{PCAP} data.
Summary information about datasets are provided in the table~\ref{tbl:datasets}.
We will denote the datasets hereby as \emph{flow} resp.  \emph{trace}.
The overal time span has been two weeks. Simulation of the malicious behavior
has been performed at duration of several days and mixed into both data sets.

The sampling frequency for \emph{flow} has been given by nature of the data. 
As it is presampled at period 5 minutes, it`s inverse, a value 0.0033Hz has been used.
In \emph{trace} data, a sample rate have been selected based on 
analysis of the inter-packet time. For the HTTP traffic it has been observed that 
that the inter-packet time for most of the packets was less than 20ms thus sample rate 
of 100Hz has been chosen according to Nyquist theorem.
 
\begin{table}[h]
    \begin{center}
        \begin{tabular}{r|cc}
        	\textbf{Dataset identification} & \textbf{flow}	&  \textbf{trace} \\ \hline
        	{Source data format} & \code{NetFlow}	&  \code{PCAP} \\ 
        	Duration [day] &  $\sim$  14 &  $\sim$  7 \\
        	Number of annotated flows [-] & 774677 & 1148\\
        	Sample rate [Hz] & 0.0033 & 100  \\
        	Window size [samples] & 284 & 200  \\ %TODO
        	Window size  [hour; second] & $\sim$ 24 & 2  \\
        \end{tabular}
    \end{center}
    \caption{\small Information about datasets}
    \label{tbl:datasets}
\end{table}


\begin{table}[h]
    \begin{center}
        \begin{tabular}{c|cc|ccc}
        	& \multicolumn{2}{ c| }{Number of samples}&&&\\
        	& \textbf{trace}& \textbf{flow}& Model	&  Negative & Positive \\ \hline
        	\emph{http} & 21249 &7183& x & x & - \\
        	\emph{vpn} & 128026 &210& - & - & x \\
        	\emph{httptunnel} & 14064 &20& - & - & x \\ %TODO
        	\emph{woodpecker} & 160054 &400& - & - & x \\
        \end{tabular}
    \end{center}
    \caption{\small Labeling of the sample}
    \label{tbl:classes}
\end{table}

\begin{table}[h]
    \begin{center}
        \begin{tabular}{c|c}
        	\textbf{Method}	& \textbf{Parameters}  \\ \hline
        	Principal Component Analysis (PCA) & number of dimensions  \\
        	Momentum (M) & list of moment functions  \\
        	Band Filter (BF) & number of bands \\
        \end{tabular}
    \end{center}
    \caption{\small Assesed methods of dimensionality reduction}
    \label{tbl:methods}
\end{table}

Table \ref{tbl:classes} contains information about classes present in dataset. A class \emph{http}
has been used to fit the models. All classes including modeled have been used in validation.

On the figures \ref{fig:roc_flow_overal} and \ref{fig:roc_trace_overal} (in appendix \ref{ch:appb})
there is a ROC curve of various methods on datasets described by table \ref{tbl:datasets}.
It depicts overal discrimative properties between positive and negative classes
as described by table \ref{tbl:classes}. 
Quantitave measures are in table \ref{tbl:auc_overal}.
Consistently better results are achieved by using methods based on PCA dimensionality reduction.

On figures \ref{fig:roc_flow_gmm}, \ref{fig:roc_trace_gmm} and table \ref{tbl:roc_gmm}
we compared how is gaussian model complexity affecting discriminative performance and we 
examined that for packet trace data, decrasing performance with increasing complexity occured.
However for \emph{netflow} dataset using a bimodal instead of unimodal model significantly improves performance. 
The plot is showing ROC curves for gaussian mixtures with varying number of components.
We decided to use unimodal distibution for packet trace data and bimodal model for flow data.

The figure \ref{fig:roc_flow_class}, \ref{fig:roc_trace_class} depicts diferences between classes. For \emph{flow} the ROC
cuves tend to by sharp and good performing in discriminating all measured classes.
Table \ref{tbl:auc_class} contains quantitative measures.

Figure \ref{fig:pca_eig} depicts an decrease of variances of eigenvectors. This trend is dependent on
data instancef of the \emph{http} class. It can be shown that taking first~2 eigenvectors is sufficient.
Plots \ref{fig:roc_flow_pca}, \ref{fig:roc_trace_pca} and table \ref{tbl:roc_pca} supports this statement. It shows
ROC curves and AUC values for varying number of eigenvectors used.


\section{Conclusion}

In present work we proposed new detection mechanism of network traffic anomalies
based on statistical analysis of the frequency components of the signal.

We were focused on detecting tunneled connections and missuse of the HTTP protocol as it is easy way to
circumvent restriction policies.


Our method estimates power spectral density of the packet process to create features.
Packet process is treated as stochastic stationary process of packet arrivals and departures.
Power spectral density has been estimated using rectangular windowing and using Fourier transform of
autocorrelation function.

Principal component analysis has been used to reduce feature space dimensionality and increase
information content in retained features.
Analysis of the eigenvalues showed that first~3 dimensions are significant. 

Finally statistical analysis based on  Mahalanobis distance with
unimodal Gaussian model has been used  to evaluate anomaly score.
Other possibilities of estimating anomaly score can involve Gaussian Mixture Models.
The score in this case, is the log-probability estimated by model.

We modeled normal behavior of the HTTP connections. The model has been fitted
by means of maximum likelihood estimator (MLE) to a testing dataset
comprising approximately two weeks of traffic. The model has been then crossvalidated using testing set 
drawn from same population using stratified k-fold crossvalidation.
For evaluation we used an ROC curves. ROC analysis showed that the method is performing well on 
the packet trace data as well as on traffic flow statistics data.

Further research is required to assess robustness in collaborative environment.
The method is intended to focus on specific behavior. In collaborative system it is intended
to be usefull in detecting  various activities by extending a models to cover different behaviors.

