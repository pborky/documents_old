
\chapter{Introduction}

An anomaly or anomalous behavior is the behavior that deviates from established ``normal'' habits.
In many cases it refers to some actionable or critical state. For that reason it has been 
researched in wide variety of domains and wide variety of anomaly detection techniques has been emerged.
For example and anomaly detection techniques are used in patients health data to detect possible
symptoms of disease. In safety critical environment an anmalous behavior can indicate
performance degradation with possible catastrophic outcomes.

In many cases the anomaly detection is related to outlier detection.
In statistics, an outliers are a data instances that are deviate from given sample in which 
they occure. Grubbs in~\cite{grubbs1969procedures} defined ``\emph{an outlying observation, 
or `outlier', is one that appears to deviate markedly from other members of the sample in which it occurs}''.
 When using outlier detection for detection of anomalies, an assumption 
that anomalies are distant from the rest of observations apply. However an outlying observation 
may be result of data acquisition error, or numerical error, or it may indicate faulty estimation of the models.

In field of computer security anomaly often refers to malicious behavior -- to behavior of some 
agent that can be harmfull and ususally unwanted at given context. 
The computer network is space that provides opportunity for malicious agents to perform activities such
as unauthorized acces, privacy violation, reduction of availability of services, fraudulent operations, etc.
Detection and prevention of most of malicious activities is important in ensuring the 
availability and reliability of the computer systems. 

Systems used to detect malicious behavior in computer security are traditionally referred to as an
\emph{intrusion detection systems} (IDS) and are divided into three main categories: a \emph{signature detection}
systems, an a\emph{nomaly detection} systems or \emph{hybrid} systems comprising both approaches. While signature
based systems depend on database of attack signatures, anomaly detection systems rely on 
models of normal behavior. Decision in a hybrid system is based on both approaches --
on a normal model as well as the malicious behavior of the attacker.

Crucial assuption in anomaly detection based intrusion detection systems is that the malicious activity
is subset of the anomalous activity. The example of such activity is the possible malicious agent
trying to penetrate information system without knowledge of its typical use. Such activity
is likely to be detected as anomalous. On contrary if similar agent uses knowledge of normal usage
it may be difficult to detect its activity.

In present work we introduce an anomaly detection technique that uses model based on spectral analysis in order
to identify a malicious behavior. Our technique is based on assumption that the attacks can introduce
irregularities at given periodic contexts. These context are reffered to as a frequency domain or
frequency spectrum. We claim that the malicious agents performing attacks can be detected as their habit 
can deviate from normal behavior. Such difference can be shown on a example of typical user that 
folow an circadian rythm that can lead in oscilations in network usage at about 24 hours.
On contrary mallware that would not follow same rythm  does not leave trails in same frequency context.
Different example could be a violation of specification of the network protocol.
In case the malicious user tries to escape security restrictions for example  by misusing a allwed
protocol to tunnel  the restricted one this can imprint a different pattern in frequency spectrum, as  
showed by \name{Chen} and \name{Hwang} in~\cite{chen2007tcp,chen2007spectral}.

\section{Related Work}

\name{Chandola et al.}~\cite{chandola2009anomaly} addressed anomaly detection in general and also 
identified various approaches and application domains.
They described methods based on \emph{classification}, \emph{clustering}, \emph{nearest neighbour}, 
\emph{statistical}, \emph{information theory} and \emph{spectral analysis}. 
They covered several application domains such as \emph{cyber-intrusion detection}, \emph{fraud detection}, 
\emph{idustrial damage detection}, \emph{sensor networks} etc. 
Their contribution with respect to our work is mainly an exact definition of anomaly detection and deep, 
structured overview of the known techniques in various application domains. In the domain of our interest --
the network intrusion detection, they depicted fact that although available data has an temporal content, 
known techniques typically do not exploit this aspect explicitly. 
The data is mostly high-dimensional with continuous as well as categorical attributes. 
The challenge faced by techniques in this domain is the changing nature of anomalies as the intruders
adapt to the existing intrusion detection solutions and the high dimensionality and high ammount of
the data. They showed that the existing methods in this domain are based on statistical analysis, classification, 
clustering, spectral decomposition and information theory.
%

\name{Patcha} and \name{Park}~\cite{patcha2007anomaly} covered cyber-intrusion domain focusing on statistical,
data-mining and machine learning techniques. They provided overview of the solutions in use and are state-of-the 
art in the cyber-intrusion detection and referenced number of research systems.
%%
%

\name{Davis} and \name{Clark}~\cite{davis2011data} focused on data preprocessing techniques for network
intrusion detection. They described \emph{dataset creation}, \emph{feature construction} and \emph{reduction}
techniques. 
In this comprehensive review they grouped a related works according to the type of features and
data preprocessing techniques they addressed. They identified aggregation of packets into flows as useful
as it enforces contextual analysis and statistical measures to detect anomalous behavior. They noticed that
packet header based approaches are not sufficient as the use of defense against attacks forced attackers 
to use different attack vectors such as crafted application data. They suggest that there is need to use 
features derived from contents of packets but as there is little research in this area they expect that
more results would emerge in future.
%
\name{Onut} and \name{Ghorbani}~\cite{onut2007feature} derived taxonomy of features used for anomaly detection.
Futhermore they introduced anomaly network intrusion detection systems which use them.
%
\name{Gogoi et al.}~\cite{gogoi2011survey} focused on comparison of specific techniques used for network anomaly
detection. They covered supervised and unsupervised approaches covering several techniques in detail, such as
statistical, signal processing, graph teoretic, clustering  or rule-based techniques.

In our study we focused on papers that uses spectral analysis. 
\name{Chen} and \name{Hwang} in~\cite{chen2007tcp,chen2007spectral}
invented an anomaly detection technique involving spectral analysis of the network traffic to analyze spectral 
characteristic of network protocols (TCP, UDP).
They were able to distinguish between different protocols using statistical methods on 
freqency spectrum of the packet arrival process. 
In adition they introduced statistical anomaly detection method to distinguish between 
legitimate and malicious TCP flows. 
An spectral characteristics of the network traffic has been researched also by
\name{X. He, et al.}~\cite{he2004spectral}.
They used an technique of spectral analysis to show the signatures of different layers of the 
network protocols.
Also time-frequency based methods has been used by \name{Salagean}~\cite{salagean2010real} or \name{Gao et al.}~\cite{gao2006anomaly}
involving a \emph{wavelet transform}. In~\cite{salagean2010real}  used a wavelet transform and higher-order statistics
to discriminate attacks from normal traffic.

We were also concerned about spectral techniques in discriminating tunneling protocols. We
observed that
\name{Wright et al.}~\cite{wright2006inferring} and \name{Dusi et al.}~\cite{dusi2009tunnel} 
investigated an detection of encrypted tunnels inside the application layer. 
They addressed the problem of bypassing an network-boundary
security inspection by encapsulating of data subject to restrictions 
(peer-to-peer, chat, e-mail and others) into protocols that are considered safe 
and necessary (HTTP, HTTPS, SSH, DNS etc.). 
\name{Est{\'e}vez-Tapiador et al.}~\cite{estevez2004measuring}  and \name{Yamada et al.} 
\cite{yamada2007intrusion} studied anomalies in encrypted traffic.
%
\name{Wright et al.}~\cite{wright2006inferring} used features derived from packet headers 
agregating packets over protocols, and time span of arrival. They counted packets in 
categories during an epoch
resulting in vector. Then they used k-nearest neighbor (kNN) and hidden Markov model 
(HMM) techniques. They constructed models for differnet kind of encrypted tunnels 
such as single- or multi-flow tunnels. They were able to infer application protocols even in
multiplexed packet flows without need of demultiplexing.
%
\name{Dusi et al.}~\cite{dusi2009tunnel} brought an statistical approach 
to detect an tunnel inside application layer.
In the paper they described diffenrent tunneling techniques and designed statistical pattern recognition 
classifier to identify them. Classification of the encrypted traffic has been also researched by \name{Ingham et al.} 
\citep{ingham2007comparing,ingham2007anomaly} or \name{Alshammari et al.}~\cite{alshammari2009machine}.
%

\section{Our Contribution}

A frequency spectrum based anomaly detection technique has been proposed by~\cite{he2004spectral,chen2007spectral,chen2007tcp}.
They analyzed properties of network protocols and also developed method for detection of network attack
causing deviation of the spectral characteristics at given context.

In our work we are going further and we use the analysis of the frequency spectrum in detection of the tunnelled connections
inside application layer and also in the detection of the mallware-like behavior. 

First we apply the detection technique at short time span (approx. 2 sec) to model a properties of the HTTP protocol and diferentiate it from
the tunneled protocols. Tunneled protocols missuse the encapsulating protocol (in this case HTTP) in order to circumvent restrictions in computer networks 
(e.g. corporate proxy server). Detection of tunneling protocols has been researched by many authors 
\cite{wright2006inferring,yamada2007intrusion,estevez2004measuring,ingham2007comparing,ingham2007anomaly,alshammari2009machine,dusi2009tunnel}, 
although none of them used frequency analysis.

Next we look at higher time span (approx. 24 hour) in order to detect an mallware that uses HTTP protocol to leak data, but it is 
assumed that is behaves differently at this time context. 

The detection methods are going to be part of system comprising different detection methods and providing agregation of partial outcomes.

\section{Organization}

This work is organized into 5 chapters. Chapter~\ref{ch:theory} brings theoretical introduction to anomaly detection and 
introduces computer network security aspects. Chapter~\ref{ch:method} provides detailed description of proposed
anomaly detection method. In chapter~\ref{ch:experiments} an implementation is introduced as well as the experimentation on
real data is depicted. Finaly, chapter~\ref{ch:conclude} concludes the work.
