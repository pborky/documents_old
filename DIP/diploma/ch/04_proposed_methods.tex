
\chapter{Proposed Method}

In this section we dervive anomaly detection method based on spectral decomposition
of time-domain signal.
Our goal is to involve statistical analysis of the frequency components of a 
time-domain signal for detection of the malicious behavior.
%\emph{tunneling protocol} %malicous traffic
%in an application layer as well as 

On the high-frequency scale our method ought to be capable of detection of the
\emph{tunneling protocols} in application layer, while on the very low frequency scale
our method is intended to be capable to detect anomalous behavior in wider time contexts.

Tunneling protocol refer to encapsulation of one network 
protocol in payload of the other. By using tuneling protocol a malicious agent can 
transfer data  belonging to prohibited protocol over a network, bypassing 
a security policies. Our hypotesis is that specific application protocol imprints 
specific pattern in a power spectral density distribution. If the tunneling protocol 
is in use, these patterns are different and anomalous behavior can be detected.

The periodicity can be analyzed on different scales, while tunneling protocol
violates normal pattern on higher frequency scales, we are also concerned about anomalous
patterns in very low frequency scales. This other patterns come up not of violation 
of the standard protocols, but of departure from the typical behavior or the habits. 
E.g. an mallware communicating with the remote server using the HTTP protocol 
does not violate the application protocol. 
But by exchanging information in regular manner, not depending on time of the day,
its behavior will be distint from the behavior of the typical web user who uses the network
in a way, periodic manner.

\name{He et al.} \cite{he2004spectral} showed that the different layers of the network
protocols inprint distinct patterns in a power spectral density distribution. 
Further work of \name{Chen} and \name{Hwang} \cite{chen2007spectral}
used the features derived %TODO use better word
from frequency spectrum to classify malicious and normal traffic.
They noticed that transport protocols (transmission control protocol --
TCP and user datagram protocol - UDP) 
have distinct power spectral density distribution. 
They exploited this property to identify low-rate denial-of-service (DoS)
attacks on TCP protocol. %TODO on TCP protocol?
%an reduction of service (RoS) attacks.
%TODO more writing about features used
\name{Wright et al.} \cite{wright2006inferring} researched methods based on 
hidden Markov models to classify different application protocols embeded in 
encrypted application layer.
They developed classification method, able to classify different 
application protocols multiplexed in single encrypted packet flow.
\name{Dusi et al.} brought an statistical approach 
to detect an tunnel inside application layer.
In  \cite{dusi2009tunnel}  they described different tunneling techniques and designed 
statistical pattern recognition classifier to identify them.

Even though it is not possible to analyze payload of particular packets in 
ecrypted connection, it is possible to observe the time of the packet transit, 
its size, direction, source and destination endpoint, etc. 
This data is denoted as packet traces and it is extracted from unencrypted 
part of the packet. 
The goal is to develop feature creation and pattern recognition method for 
the network packet traces involving spectral analysis.
The method is supposed to detect tunneling protocols only by observation 
of the packet traces.

\section{Data Collection}

The input data for our method consists of a timestamped \emph{packet traces} or a
timestamped \emph{traffic flow} statistics. 

Both can be obtained by capturing
network packet headers. While first one stores each packet (or the packet header)
as a single record,
other one contains total count of the packets and ammount of bytes for 
related sequence of the packets. Relation between packets is determined depending 
on the traffic flow protocol. E.g. for Transmission Control Protocol (TCP) relation is 
determined using following information: capturing interface, source and destination IP address,
source and destination port. 
%
%or Cisco standard \emph{NetFlow}%
%\footnote{
%Cisco standard definest one record (a flow) in Netflow format as a unidirectional sequence of
%packets that share following values: ingress interface; source IP address; destination 
%IP address; IP protocol; source port for UDP or TCP, 0 for other protocols;
%destination port for UDP or TCP, type and code for ICMP, or 0 for other protocols;
%type of service (TOS).
%} 
%protocol. As NetFlow datasets contains only statistics of particular network packet flows, 
%it can be generated from the packet trace data. In practice the NetFlow data is generated by 
%network active components as firewals, routers or swithes. 
%

Due to differences in described formats, we extract following attributes 
from packet trace or traffic flow data in our experiments, 
to allow usage of unified processing methods:
\begin{itemize}
	\item packet or packet flow \emph{timestamp}, i.e. the time when the (first) packet passed
	trought the capturing gateway,
	\item \emph{flow 5-tuple} -- i.e. tranmission \emph{protocol} speciffication and identification of 
	source and destination endpoint
	(e.g. for the TCP or the UDP protocols it is \emph{source address} and \emph{port}, 
	\emph{destination address}  and \emph{port}),
	\item \emph{size} of packet`s payload or \emph{total size} of the packets in case of flow data,
	\item \emph{count} of the packets (for packet trace data it is always one),
	\item \emph{direction}%
	\footnote{%
		We embed \emph{direction information} 
		into the \emph{size} parameter using negative size  
		if packets travel from destination to source and otherwise positive.%
	} %
	with respect to initial packet whithin given flow,
\end{itemize}

We define source endpoint as the endpoint that initiates connection and the destination edpoint 
as the endpoint that  accepts the connection. %TODO what?
Inbound and outbound packets are distinguished
by direction attribude. 

We further require that traffic flow data are allways captured periodically in defined time span.
We can look at this process as a sampling process. We consider the flow capturing period
is lower bound for the sampling period. This can of course cause alliasing as the original
signal contains higher frequency components than the sampling frequency induced by data capturing 
process. For packet trace data there is no upper bound on the sampling frequency induced by 
capturing process, but higher sampling frequency raises the memory and processig time requirements.

\subsection{Training Dataset}

%TODO move to section 5
A \code{tcpdump} software \cite{jacobson2009tcpdump}  has been used to 
capture packet data in experimental testbed. 
Testbed consist of three or more peers \emph{A}, \emph{B}, \emph{C}, \emph{D}  etc. 
Peer \emph{A} (a client) is connected with each other using secure tunnel 
to the destination TCP port 443 (destination is on the peers \emph{B},  \emph{C}, ...).
Peer \emph{B} serves virtual private network service and provides network address
translation from the tunnel to the internet.
Peers \emph{C}, \emph{D}, etc. are able to serve single service on the end of tunnel 
(such as HTTP server, telnet, file tranfer service, VOIP, ...).
The traffic tunneled trough peer \emph{B} consist varying number of connections 
and different protocols,
while the others are tunneling single application protocol.

The peer \emph{A} generates traffic to other peers simultaneously, 
mimicking typical behavior of the client of particular application protocol 
(i.e. web browser).
%In addition connections to HTTPS servers are performed from peer \emph{A} directly.
Packet capturing is set up on peer \emph{A} intercepting and storing every packet.
Captured packet flows are then labeled using name of known the application service. 

\subsection{Evaluation Dataset%
\footnote{Evaluation dataset is used during 
assesment of the method on real data.
As we have some old packet trace data, we want to somehow compare it with training samples.
My idea is to prove that exctracted features are epoch invariant by means
statistical goodnes-of-fit test.
On the other hand older protocols can behave in diferent way and thus the 
normal models derived from current data would be
underestimated during assesment proces.
}%
}
The data used for evaluation of proposed method, consists of public packet
traces obtained at....\emph{(to be discussed)}
mixed with simulated and anotated packet traces captured in our testbed.
Before mixing, statistical test has been performed.
The test procedure first extracts and tansforms features
(denoted here as $\vec{f}$) by using our proposed method 
and then compares distribution of features with respect to 
the label -- $Pr\left[ \vec{f} | label \right] $.

Labeling of the obtained dataset is obtained by inferring application 
protocol using destination port.
Since the public dataset can contain malicious connections, 
the result of this statistical test will underestimate the real fitness.
%TODO why?

\section{Feature Creation and Pattern Definition}

\subsection{Stochastic process}
% TODO some flesh here

For a specified flow $f$ the packet arrival process $x_f\left[t\right]$ 
(or simply \emph{packet process})  is defined as a count of packet arrivals 
at given timespan $I = \left\langle \frac{t}{s}, \frac{t+1}{s} \right)$:
\begin{equation}\label{packetprocess}
\begin{split}
	 x_f\left[t\right] = \left| 
	\left\lbrace p : f = flow(p) \wedge time(p) \in I \right\rbrace \right|\\
	\forall t \in \mathbb{N}\, ,
\end{split}
\end{equation}
where $s$ is the sample rate, function $flow(p)$  yields the \emph{flow} 5-tuple 
and function $time(p)$  yields the \emph{timestamp} of given packet $p$. 
We can also define packet process for inbound and outbound flows separately:
\begin{equation}\label{xpacketprocess}
\begin{split}
  x_{f,d}\left[t\right] = \left| 
  \left\lbrace p : flow(p) = f \wedge dir(p) = d \wedge time(p) \in I  \right\rbrace \right|\\
  \forall t \in \mathbb{N}\, ,
\end{split}
\end{equation}%
where $dir(p)$ yields a direction of the packet. 

Note that we will refer to single flow and to avoid confusion
we will denote packet process as $x$ and the inbound and
outbound packet processes as $i$ and $o$  in equations. 
%\footnote{
%This is very important formula, we need to focus on it; according 
%to \emph{Dusi et al.} \cite{dusi2009tunnel}
%zero-length packets are ulikely to be induced by application, 
%so we can exclude them here; in addition
%they extract incomming and outgoing  stream separately -- 
%I think it is good idea to work with in- and out- streams separately 
%and compute cross-correlation function instead of auto-correlation function.
%The idea of usage I/O cross-correlation function $R_{io}\left(\tau\right)$ 
%instead of auto-correlation is that $R_{io}\left(\tau\right)$ (at the specific 
%time-lag $\tau$) would enforce the typical request-response round-trip. 
%Questionable is, how to physically interpret the resulting spectral components 
%and if the Wiener-Kitchin theorem is applicable.
%}


\subsection{Spectral density estimation}
% TODO some flesh here -  more about wiener-kitchine theorem and sampling theorem

According to a Wiener–Khinchine %TODO reference
theorem the power spectral density $S_{xx}(f)$ (PSD) of the wide-sense stationary 
stochastic process is obtained by application of discrete-time 
Fourier transform $\mathcal{F}_{\cdot}(\omega)$ on autocorrelation function 
of the packet process $R_{xx}\left[\tau\right]$:

\begin{equation}\label{eq:corr}
R_{xx}\left[\tau\right] = E[x\left[t\right]x\left[t+\tau\right]]\, , 
\end{equation}

\begin{equation}\label{eq:psd}
\begin{split}
S_{xx}(\omega) = \mathcal{F}_{R_{xx}}\left(\omega\right) = \sum_{\tau=-\infty}^{\infty} 
\left( R_{xx}\left[\tau\right] \exp\left( -\imath \omega\tau \right)\right) \\ 
\forall f \in \left\langle -\frac{s}{2},\frac{s}{2} \right\rangle\, , 
\end{split}
\end{equation}

where $\tau$ is the time-lag, $E\left[\cdot\right]$ is expected value of a random variable, $\imath$
is the imaginary unit and $\omega$ is the angular frequency $\omega= 2\pi f$. 
%The autocorrelation function is capable of enforcing periodicity. %TODO really?

Analyzing the inbound and outboud packet process can lead to definition of the cross spectral
density\cite{penny2000signal}.
As the power spectral density is Fourier tranform of the autocorrelation function of the 
stochastic process cross spectral density is the fourier transform is the fourier transform
of the cross-correlation function of two stochastic processes. The cross spectral density is
complex as because the cross-correlation function is not symmetric. 
For inbound and outbound packet process we define cross-correlation function $R_{io}$ 
as well as the cross-spectral density $S_{io}$ as follos:

\begin{equation}\label{eq:xcorr}
R_{io}\left[\tau\right] = E[x_{in}\left[t\right]x_{out}\left[t+\tau\right]]\, , 
\end{equation}

\begin{equation}\label{eq:xpsd}
\begin{split}
S_{io}(\omega) = \mathcal{F}_{R_{io}}\left(\omega\right) = \sum_{\tau=-\infty}^{\infty} 
\left( R_{io}\left[\tau\right] \exp\left( -\imath \omega\tau \right)\right) \\ 
\forall f \in \left\langle -\frac{s}{2},\frac{s}{2} \right\rangle\, , 
\end{split}
\end{equation}

Equations (\ref{eq:corr}), (\ref{eq:psd}), (\ref{eq:xcorr}) and (\ref{eq:xpsd}) 
hold under assumption that packet process is \emph{wide-sense stationary random proces}.
This asssumption seems to be false for infinite time span in network traffic.
Furthermore the time span is usually limited to finite number of samples.
For practical reasons we involved an \emph{windowing function} $w(n)$ and 
\emph{discrete Fourier transform} instead of discrete-time Fourier transform. 
The simplest windowing function -- a rectangular windowing is defined as follows:
\begin{equation}
w(n) = \left\lbrace \begin{array}{l} 
1, \mbox{ if } n\in \left\langle 0, M \right) \\ 
0, \mbox{ otherwise} \end{array}\right. \,.
\end{equation}
Windowing function is nonzero inside specified interval $\left\langle 0, M \right)$ 
otherwise it is zero. 
Parameter  $M$ is length of sub-sequence selected from packet arrival proces. 
If the parameter $M$ is too high the packet proces is unlikely to be stationary, 
on the other hand selecting too low value causes spectral leakage, i.e. the energy 
of the main lobe of a spectral response "leaks" to the sidelobes distorting the 
spectral responses \cite{kay1981spectrum} (figure \ref{fig:leakage_rect}). 

\begin{figure}[h!]%
  \centering
        \begin{subfigure}[b]{0.5\textwidth}
                \centering
                \includegraphics[width=\textwidth]{img/leakage_rect_1000}
                \caption{$n_{sampl}=1000$}
                \label{fig:leakage_rect_1000}
        \end{subfigure}%
        ~ \begin{subfigure}[b]{0.5\textwidth}
                \centering
                \includegraphics[width=\textwidth]{img/leakage_rect_25}
                \caption{$n_{sampl}=25$}
                \label{fig:leakage_rect_25}
        \end{subfigure}%
  \caption{\small Portion of the discrete-time Fourier transform (DTFT) of a rectangularly
  windowed sinusoid. The unit on x-axis is an Discrete fourier transform (DFT) bin.
  The red dots show values returned by DFT. It is visible that for decreased widows size
  the leakage is high.}
  \label{fig:leakage_rect}
\end{figure}

the spectrum is sensitive 
to transient phenomena on the network. %TODO why? 

We iteratively apply windowing function to the whole sequence generating 
non-overlapping adjancent sequence of windows. 
We identify the particular window in this sequence with upper index -- 
e.g. $S_{xx}^i$ is a power spectral density function of an $i$-th window.
Thus, we rewrite equations (\ref{eq:corr}) and (\ref{eq:psd}) for non-overlapping 
windows as follows:
\begin{equation}\label{eq:corr2}
R_{xx}^i\left[m\right] = \frac{1}{M} \sum_{t=0}^{M} x\left[t+iM\right]x\left[t+m+iM\right] \, , 
\end{equation}
\begin{equation}\label{eq:psd2}
\begin{split}
S_{xx}^i(k) = \mathcal{F}_{R_{xx}^i}\left(k\right) = \sum_{m=0}^{M-1}
\left( R_{xx}^i \left[m\right] w(m) \exp\left( -\imath 2\pi m\frac{k}{M} \right)\right)\\
\forall k \in \left\{ 0,1,2,...,M-1 \right\}\, . 
\end{split}
\end{equation}
Note that the windowing function is inherent in equations (\ref{eq:corr2}) and (\ref{eq:psd2})
by using limitted ranges of sumation, and domain of definition of the power spectral density.

By involving windowing function we introduced
\emph{spectral leakge}. Use of apodization function, e.g. \emph{Hann} %TODO explain 
(see equation \ref{eq:hann}),
%or \emph{Hamming} (\ref{eq:hamm}) 
could be appropiate in decreasing leakage.
\begin{equation}\label{eq:hann}
w(n) = \left\lbrace \begin{array}{l} 
0.5\left(1 - \cos \left ( \frac{2 \pi n}{N-1} \right) \right), \mbox{ if } n\in \left\langle 
0, M \right) \\ 0, \mbox{ otherwise} \end{array}\right. \,.
\end{equation}
%\begin{equation}\label{eq:hamm}
%w(n) = \left\lbrace \begin{array}{l} 
%0.54 - 0.46\cos \left ( \frac{2\pi n}{N-1} \right), \mbox{ if } n\in \left\langle 0, M \right) \\
% 0, \mbox{ otherwise} \end{array}\right. \,.
%\end{equation}

The sampling rate $s$ must be selected according to the Nyquist theorem. 
Too low value entails aliasing%
\footnote{The aliasing is caused by folding of the frequencies above Nyquist frequency
$\frac{s}{2}$ symmetrically below this frequency. Thus this two frequencies are undistinguishable.
To properly reconstruct the signal that contain no frequency higher than $f_{max}$ 
the sample rate is bounded by $s > 2f_{max}$.}%
, while too high value incurs data storage and processing overhead. 

%\subsection{Pattern definition}
By the application of the Fourier transform original features has been mapped into new space.
We denote features in new space as \emph{spectral components} (or \emph{frequency components}).
Temporal context of the features has been altered. 
In new feature space the temporal context is determined by sequence
of the detection windows of size $M$. 
At the level of the detection window a temporal notion is decomposed 
into tuple of temporal functions parametrised by \emph{spectral components}.
Although temporal aspect is still present it has not been used in further analysis.
Anomalies in new feature space are thus regarded to as \emph{point anomalies}. %TODO ?
%The rationale is that the new features decomppsed desired temporal aspects 
%of the original sequential data into declarative form.

There are few parameters affecting quality of the  features: %TODO what?
the smaple rate $s$ and the window length $M$. 
In addition, \emph{spectral components} can be subject to further feature extraction 
comprising combination of extisting features or discarding irrelevant, 
redundant  or noisy%
\footnote{In case of low signal-to-noise ratio, a feature is typically not usefull 
for discriminative outcome.}%
ones based on the domain knowledge (e.g. sum of a lower resp. a upper half of the 
spectral components resulting in two features a low- and a high-frequency power densities;
or retaining specific spectral features known for its relation to 
periodic stochatic processes to be in research interest).

This aspects and process of seeking of proper parameters are subject of further 
research and they are discussed in \emph{chapter \ref{sec:experiments}}.

\section{Anomalous Pattern Recognition}

\section{Assessment and Interpretation of Results}
