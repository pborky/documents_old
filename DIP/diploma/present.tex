%%*************************************************************************
%%
%% Network Anomaly Detection by Means of Spectral Analysis
%% V1.2
%% 2013/01/03
%% by Peter Boraros
%% See http://www.pborky.sk/contact for current contact information.
%%
%%*************************************************************************
%%
%% Legal Notice:
%%
%% This code is offered as-is without any warranty either expressed or
%% implied; without even the implied warranty of MERCHANTABILITY or
%% FITNESS FOR A PARTICULAR PURPOSE! 
%% User assumes all risk.
%%
%% This work by Peter Boraros is licensed under a 
%% Creative Commons Attribution-NonCommercial-ShareAlike 3.0 Unported License.
%% http://creativecommons.org/licenses/by-nc-sa/3.0/
%%
%%*************************************************************************

\documentclass{beamer}

\usepackage[utf8x]{inputenc}


\usepackage{array}
\usepackage{longtable}
\usepackage{varioref}
\usepackage{wrapfig}
\usepackage{fancybox}
\usepackage{calc}
\usepackage{framed}
\usepackage{url}
\usepackage{amssymb,amsmath}

\usepackage{graphicx}
%\DeclareGraphicsExtensions{.pdf,.png,.jpg}
\renewcommand{\figurename}{Fig}
\usepackage{caption}
\usepackage{subcaption}
 
\usepackage{multicol}
\usepackage{multirow}
\usepackage{color}
\usepackage{soul}
%<-------------------------------headers----------------------------------------->
\usepackage{headers}
%<-------------------------------společná nastavení------------------------------>
\usepackage{hyperref}
%\usepackage[hyphenbreaks]{breakurl}
\usepackage[numbers,sort&compress]{natbib} %balíček pro citace literatury  
\usepackage{hypernat}%interakce mezi hyperref a natbib
\hypersetup{   % Nastavení polí PDF dokumentu 
pdftitle={\Title},%   
pdfauthor={\AuthorName},%  
pdfsubject={\Subject},%   
pdfkeywords={\Keywords}%
}

%% include whole pdfpages (only if using pdflatex)
\ifpdf
\usepackage[final]{pdfpages}
\fi

%<------------------------------user macros----------------------------------------->
\usepackage{packages/mypkg}

\usepackage[T1]{fontenc}
%\usepackage[utf8]{inputenc}

\usetheme{Warsaw}

\title{\Title}
\subtitle{\Subject}
\author{\AuthorName}
\date{January 24, 2013}
%\institute{}
% for bibtex
\newcommand{\newblock}{}

\begin{document}


\begin{frame}
\titlepage
\end{frame}


\begin{frame}{Outline}
\tableofcontents%[pausesections]
\end{frame}

\section{Introduction}

\subsection{Motivation}

\begin{frame}{Assignment}
\begin{center}
The goal of this thesis is to design and implement an anomaly detection method 
based on spectral analysis.
In particular, use the proposed method to distinguish normal and anomalous 
network traffic and evaluate the
properties of this method on real network traffic data.
\end{center}
\end{frame}

\begin{frame}{Why spectral analysis?}
\begin{itemize}
	\item<1-> study the periodic behavior of malicious vs. legitimate agents
	\item<2-> distinguish the legitimate behavior from malicious 
	based on statistical properties of frequency spectrum
	\item<3-> assumption that statistical anomaly is induced by malicious behavior, 
	and malicious behavior induces anomalous observation
\end{itemize}
\end{frame}

\subsection{Input data}

\begin{frame}{Input data}
\begin{itemize}
	\item<1-> observation of sending and receiving packets
	\item<2-> temporal context of the packets
	\begin{itemize}
		\item<2-> timestamp
	\end{itemize}
	\item<3-> "spatial" context of the packets
	\begin{itemize}
		\item<3-> flow identification: \alert<4->{source \& destination endpoint, protocol}
	\end{itemize}
\end{itemize}
\end{frame}

\begin{frame}{Input data -- spatial context (identification of flows)}
\begin{itemize}
	\item<2-> we are interested in HTTP(S) $\Rightarrow$ 
	\begin{itemize}
		\item<3-> always TCP
		\item<4-> not allways reuses existing socket
	\end{itemize}	
	\item<1,2|only@1,2> {src. IP addr. \& port}, {dest. IP addr. \& port},
					protocol (TCP, UDP, ICMP, ...)
	\item<3|only@3> {src. IP addr. \& port}, {dest. IP addr. \& port},
					\alert{\st{protocol (TCP, UDP, ICMP, ...)}}
	\item<4|only@4> {src. IP addr. \& \alert{\st{port}}}, {dest. IP addr. \& port}
	\item<5|only@5> {src. IP addr.}, {dest. IP addr. \& port}
	\item<6|only@6> {\color{red} src. IP addr} (client), {dest. IP addr \& port}
	\item<7|only@7> {\color{red} src. IP addr} (client), {\color{green} dest. IP addr \& port} (server)
\end{itemize}
\begin{center}
	\begin{figure}[h] %
		\includegraphics<1,2,3,4,5>[width=60mm]{img/cloud}
		\includegraphics<6>[width=60mm]{img/cloud1}
		\includegraphics<7>[width=60mm]{img/cloud2}
		\label{fit_et}
	\end{figure}
\end{center}
\end{frame}


\begin{frame}{Input data -- packet matrix}
\begin{itemize}
	\item<2-> \textbf{libpcap}\footnote<2-|only@2->{open source library used by: wireshark, tcpdump 
	\cite{jacobson2009tcpdump}, ...} 
		-- captures every single packet
	\begin{itemize}
		\item<2-> trucation of the packet payload
	\end{itemize}
	\item<3-> \textbf{netflow}\footnote<3-|only@2->{protocol introduced by Cisco in \cite{claise2007rfc}} 
		-- captures statistics about flows
	\begin{itemize}
		\item<3-> only packet statistics are captured at given sampling interval
	\end{itemize}
\end{itemize}
\begin{center}
{\small
\begin{tabular}{c|c|c|c|c}
\multicolumn{1}{c|}{\textbf{spatial}} &
\multicolumn{1}{c|}{\textbf{temporal}} &
\multicolumn{2}{c}{\textbf{behavioral}}\\
$\wrapan{\text{src.IP, dst.IP, dst.port}}$ & timestamp [ms] & {bytes} & {packets} \\
\hline
$\left\langle 167797783 , 3222593812 , 80 \right\rangle$ & 1353685821602660 & \only<2>{406}\only<3->{5450} & \only<2>{1}\only<3->{10} \\
$\left\langle 179580632 , 179580226 , 3128 \right\rangle$ & 1353685824241826 & \only<2>{103}\only<3->{23900} & \only<2>{1}\only<3->{187} \\
$\left\langle 167797783 , 2915182441 , 443 \right\rangle$ & 1353685824242867 & \only<2>{898}\only<3->{12569} & \only<2>{1}\only<3->{66} \\
$\left\langle 167797783 , 2915190380 , 993 \right\rangle$ & 1353685824351661 & \only<2>{37}\only<3->{102} & \only<2>{1}\only<3->{3} \\
1,
\end{tabular}
}
\end{center}
\end{frame}

\section{Proposed method}

\subsection{Stochastic packet process}

\begin{frame}{Stochastic packet process -- sampling}
\begin{block}{Packet process}
%\begin{center}
For given flow $f$ we define a packet process $x_f\wrapsq{t}$ as
%\end{center}
\begin{equation}\label{packetprocess}
\begin{split}
	 x_f\left[t\right] = \left| 
	\left\lbrace p : f = flow(p) \wedge time(p) \in I_t \right\rbrace \right|\\
	\forall t \in \mathbb{N}\, ,
\end{split}
\end{equation}
%\begin{center}
where $I_t = \left\langle \frac{t}{s}, \frac{t+1}{s} \right)$  is timespan,
$s$ is the sample rate, function $flow(p)$  yields the \emph{flow identification}
and function $time(p)$  yields the \emph{timestamp} of given packet $p$. 
%\end{center}
\end{block}
\end{frame}

\subsection{Power spectral density}
\begin{frame}{Power spectral density -- Wiener–Khinchine theorem}
\begin{block}{Wiener–Khinchine theorem}
%\begin{center}
Power spectral density $S_{xx}(\omega)$ of the wide-sense stationary 
stochastic process is obtained by application of discrete-time 
Fourier transform $\mathcal{F}_{\cdot}(\omega)$ on autocorrelation function 
of the packet process $R_{xx}\left[\tau\right]$:
%\end{center}
\begin{equation}\label{eq:corr}
R_{xx}\left[\tau\right] = E[x\left[t\right]x\left[t+\tau\right]]\, , 
\end{equation}
\begin{equation}\label{eq:psd}
\begin{split}
S_{xx}(\omega) = \mathcal{F}_{R_{xx}}\left(\omega\right) = \sum_{\tau=-\infty}^{\infty} 
\left( R_{xx}\left[\tau\right] \exp\left( -\imath \omega\tau \right)\right) \\ 
\forall \omega \in \left\langle -\pi s,\pi s \right\rangle\, , 
\end{split}
\end{equation}
%\begin{center}
where $\tau$ is the time-lag, $E\left[\cdot\right]$ is expected value of a random variable, $\imath$
is the imaginary unit and $\omega$ is the angular frequency $\omega= 2\pi f$. 
%\end{center}
\end{block}
\end{frame}


\begin{frame}{Power spectral density -- windowing}
\begin{itemize}
	\item<1-7|only@1-7> assumption of \emph{wide-sense stationary stochastic proces}
	\item<2-8|only@1-8> seems to be false for \textbf{infinite} time span
	\item<3-9|only@1-9> time span is naturaly limited to finite number of samples
	\item<4-> $\Rightarrow$ \textbf{windowing function} $w(n)$
\begin{equation}
w(n) = \left\lbrace \begin{array}{l} 
1, \mbox{ if } n\in \left\langle 0, M \right) \\ 
0, \mbox{ otherwise} \end{array}\right. \,
\end{equation}
	\item<5->  $M$ is length of sub-sequence selected from packet process
	\begin{itemize}
		\item<6-> if $M$ is too high the packet proces is unlikely to be stationary
		\item<7-> selecting too low value causes spectral leakage i.e. the energy 
of the main lobe of a spectral response "leaks" to the sidelobes distorting the 
spectral responses
	\end{itemize}
	\item<8-> iteratively apply windowing function to the whole sequence
	\item<9-> compute power spectral density $S_{xx}^\wrappa{i}\wrappa{k}$ in each iteration 
\end{itemize}
\end{frame}


\begin{frame}{Power spectral density -- windowing}
\begin{block}{Power spectral density with windowing}
Power spectral density $S_{xx}^\wrappa{i}\wrappa{k}$ in $i$-th iteration:
\begin{equation}\label{eq:corr2}
R_{xx}^\wrappa{i}\left[m\right] = \frac{1}{M} \sum_{t=0}^{M}
 x\left[t+iM\right]x\left[t+m+iM\right] \, , 
\end{equation}
\begin{equation}\label{eq:psd2}
\begin{split}
S_{xx}^\wrappa{i}(k) = \mathcal{F}_{R_{xx}^\wrappa{i}}\left(k\right) = \sum_{m=0}^{M-1}
\left( R_{xx}^\wrappa{i} \left[m\right] w(m) \exp\left( -\imath 2\pi m\frac{k}{M} \right)\right)\\
\forall k \in \left\{ 0,1,2,...,M-1 \right\}\, . 
\end{split}
\end{equation}
\end{block}
\begin{itemize}
\item<2> sample set: $\mathbf{X} = \wrapsq{S_{xx}^\wrappa{i}\wrappa{k}}_{i\in \mathbb{N},k \in \left\{ 0,1,2,...,M-1 \right\}}$
\end{itemize}
\end{frame}

\subsection{Feature extraction}
\begin{frame}{Feature extraction}
\begin{itemize}
	\item<1-> sample set: $\mathbf{X} = \wrapsq{S_{xx}^\wrappa{i}\wrappa{k}}_{i\in \mathbb{N},k \in \left\{ 0,1,2,...,M-1 \right\}}$
	\item<2-> data instance is one row in matrix $\mathbf{X}$
	\item<3-> decrease dimensionality of data instances
	\begin{itemize}
		\item<4-> frequency band filters (BF) $\mathbf{X_t} = \mathbf{A} \mathbf{X}$
		\item<5-> moments of data instances (M) $\overline{\mathbf{M}} = \left[ \frac{1}{n} \sum_{i=1}^n\mathbf{x}_i^k\right]_{k=1\dots m}$ \\
		
		\item<6-> \alert<11>{principal component analysis (PCA)}
		\begin{itemize}
			\item<7-> retaining first $m$ principal components
			\item<8-> first $m$ eigenvectors $\mathbf{E} = \left[ e_i^\top \right]_{i=1}^m$
			\item<9-> transformed sample $\mathbf{X_t} = \mathbf{E} \mathbf{X}$
			\item<10-> explained variance $q_m = \frac{\sum_{i=1}^m\lambda_i}{\sum_{j=1}^n\lambda_j} $
		\end{itemize}
	\end{itemize}
\end{itemize}
\end{frame}

\subsection{Model}
\begin{frame}{Model}
\begin{itemize}
	\item<1-> Gaussian distribution $f_{\mathbf x}(x_1,\ldots,x_m)\, =
\frac{1}{(2\pi)^{m/2}|\boldsymbol\Sigma|^{1/2}}
\exp\left(-\frac{1}{2}({\mathbf x}-{\boldsymbol\mu})^T{\boldsymbol\Sigma}^{-1}({\mathbf x}-{\boldsymbol\mu})
\right),
$
	\begin{itemize}
	\item<2-> assuming central limit theorem (i.i.d. instances)
	\item<3-> maximum-likelihood estimator (MLE) of the covariance matrix
	\item<4-> use of diagonal  covariance matrix in case of PCA is appropiate
	\begin{itemize}
		\item<4-> number of parameters decreases from $m^2+m\rightarrow 2m$
		\item<5-> $\Rightarrow$ decreases also sample cardinality needed
	\end{itemize}
	\item<6-> \alert<8>{Mahalanobis test statistic $y^2 = \wrappa{\mathbf{x} - \overline{\mathbf{x}}}{\boldsymbol\Sigma}^{-1}\wrappa{\mathbf{x} - \overline{\mathbf{x}}}^\top$}
	\item<7-> Grubbs test using Student`s $t$-distribution ${y^2} > y^2_\Theta$ where $ y^2_\Theta = \frac{N-1}{\sqrt{N}}\sqrt{\frac{t^2_{0.5\alpha/N,N-2}}{N-2+t^2_{0.5\alpha/N,N-2}}}$ at significance level $t^2_{0.5\alpha/N,N-2}$
	\end{itemize}
	\item<9-> Gausian mixture model (GMM)
	\begin{itemize}
	\item<10-> equally weighted
	\item<11-> ``raw score`` is the log-probability under GMM model
	\end{itemize}
\end{itemize}
\end{frame}

\section{Experiments}
\subsection{Evaluation}

\begin{frame}{Evaluation -- ROC analysis}


\begin{itemize}
	\item<1-5|only@1-5> Receiver Operator Characteristic
	\item<2-6,8|only@2-6,8> \alert<8>{requires labeled training and testing samples}
	\item<3-> plot of the true positive rate $r_{tp}$ vs. false positive rate $r_{fp}$
	\item<4-5|only@4-5>[]
{\small
\begin{center}
\begin{table}[h]
    \begin{center}
        \begin{tabular}{c|cc}
        	%& Sample is anomalous & Sample is normal \\ \hline
        	& $H_0$ is true (P)	& $H_0$ is false (N) \\ \hline
        	reject $H_0$ & false negative (FN) & true negative (TN) \\
        	fail to reject $H_0$ & true positive (TP) & false positive (FP) \\ %\hline
        \end{tabular}
    \end{center}
    \label{tbl:hypo}
\end{table}
\end{center}
}
	\item<4-5|only@4-5>[] $H_0$ refutation it test statistic $y^2 > y^2_\Theta$
	\item<5|only@5> $r_{tp} = \frac{{tp}}{{tp}+{fn}}  $ and $r_{fp} = \frac{{fp}}{{fp}+{tn}}$
	\item<6-|only@6->[]
\begin{center}
	\begin{figure}[h] %
		\includegraphics[width=60mm]{img/roc_space}
		\label{roc_space}
	\end{figure}
\end{center}
	
	\item<6-|only@6-> both $r_{tp}$ and $r_{fp}$ are functions of threshold $y^2_\Theta$
	\item<7|only@7> measure of performance Area Under Curve -- AUC statistic
\end{itemize}

\end{frame}

\begin{frame}{Evaluation -- stratified k-fold cross-validation}

\begin{itemize}
	\item<1->[]
\begin{center}
	\begin{figure}[h] %
		\includegraphics[width=60mm]{img/crossval}
		\label{crossval}
	\end{figure}
\end{center}
	\item<1-> divide sample to disjoint subsamples at random
	\item<2-> stratified -- the distribution of labels is uniform over the folds
	\item<3-> measure of performance \only<4>{ -- sample mean of the AUC statistics }
	\only<4->{$\overline{\mu}_{AUC} = \frac{1}{k}\sum_{i=1}^k AUC_i$}
	\begin{itemize}
		\item<5-> biased measure (underestimate)
	\end{itemize}
	\item<6-> ``measure of robustness`` \only<7>{ -- standard deviation of the sample }
	\only<7->{$s_{AUC}^2 = \sqrt {\frac 1n \sum_{i=1}^k \left(AUC_i - \overline{\mu}_{AUC} \right)^ 2}$}
	\begin{itemize}
		\item<8-> lack of labeled data and misslabeled data
	\end{itemize}
\end{itemize}
\end{frame}


\subsection{Sample}

\begin{frame}{Input data}
 \begin{itemize}
 	\item<1-> non-public captured data (packet trace \& netflow)
	\item<2-> simulated attacks (packet trace)
	\begin{itemize}
		\item<3-> \emph{woodpecker} -- malware
		\item<4-> \emph{OpenVpn} -- virtual private network using HTTP connect
		\item<5-> \emph{httptunnel} -- encapsulate payload in HTTP requests
	\end{itemize}
	\item<6-> brief of used datasets 
 {\small
\begin{table}[h]
    \begin{center}
        \begin{tabular}{r|cc}
        	\textbf{Dataset identification} & \textbf{flow}	&  \textbf{trace} \\ \hline
        	{Source data format} & \code{NetFlow}	&  \code{PCAP} \\ 
        	Duration [day] &  $\sim$  14 &  $\sim$  7 \\
        	Number of labeled flows [-] & 774677 & 1148\\
        	Sample rate [Hz] & 0.0033 & 100  \\
        	Window size [samples] & 284 & 200  \\ %TODO
        	Window size  [hour; second] & $\sim$ 24 & 2  \\
        \end{tabular}
    \end{center}
    \label{tbl:datasets}
\end{table}
}
\end{itemize}
\end{frame}

\begin{frame}{Sample}
\begin{itemize}
	\item<1-> sample: 
	$\mathbf{X} = \wrapsq{S_{xx}^\wrappa{i}\wrappa{k}}_{i\in \mathbb{N},k \in \left\{ 0,1,2,...,M-1 \right\}}$
	\item<2-> brief of sample labeling
{\small
\begin{table}[h]
    \begin{center}
        \begin{tabular}{c|cc|ccc}
        	& \multicolumn{2}{ c| }{Number of instances}&&&\\
        	\textbf{annotaion} & \textbf{trace}& \textbf{flow}& model	&  negative & positive \\ \hline
        	\emph{http} & 21249 &7183& x & x & - \\
        	\emph{vpn} & 128026 &210& - & - & x \\
        	\emph{httptunnel} & 14064 &20& - & - & x \\ %TODO
        	\emph{woodpecker} & 160054 &400& - & - & x \\
        \end{tabular}
    \end{center}
    \label{tbl:classes}
\end{table}
}
\end{itemize}
\end{frame}

\subsection{Implementation}


\begin{frame}{Experimental software PyNfSA}
\begin{itemize}
	\item<1-> integrates packet manipulation libraries (libpcap \cite{jacobson2009libpcap}, scapy) ...
	\item<2->[] ... with  scientific tools and databases (numpy \cite{oliphant2006numpy}, 
	scikits \cite{pedregosa2011scikit}, pytables -- an array database based on HDF5 toolkit)
	\item<3-> uses HDF5 array database to store data (same as Matlab)
	\begin{itemize}
		\item<4-> converts RAW packet, netflow data to Matlab-compatible format
		\item<5-> enables batch processing, stores partial results
	\end{itemize}	
	\item<6-> support for online packet captures 
	\begin{itemize}
		\item<7->	requires escalation of privileges
	\end{itemize}
	\item<8-> fully functional on unix-like operating systems
	\item<9->  suppoosed to work on other operating systems 
	\item<10-> available at \cite{boraros2012pynfsa} under open-source license
\end{itemize}
\end{frame}

\subsection{Measurements}

\begin{frame}{Dimesionality reduction}
\begin{itemize}
	\item Sample mean and standard deviation of the AUC ROC when discriminating 
    \emph{http} from \emph{woodpecker}, \emph{httptunnel} and \emph{vpn}.
    Comparison of different dimensionality reduction techniques.
{\small
\begin{table}[h]
    \begin{center}
        \begin{tabular}{c|cc|cc}
        \multirow{2}{*}{\textbf{Method}}  &\multicolumn{2}{ c| }{\emph{flow}} &\multicolumn{2}{ c }{\emph{trace}}\\
         & $\overline{AUC}$ &  \alert<2>{$s_{AUC}$} & $\overline{AUC}$ & $s_{AUC}$ \\ \hline

M(var,curt,skew) & 0.479 & \alert<2>{0.011}  & 0.643 & 0.004\\ \hline
\alert<3->{PCA(dim=3)} & \alert<3->{0.760} & \alert<2->{0.085} & \alert<3->{0.779} & \alert<3->{0.002}\\ \hline
PCA(dim=1) & 0.616 & \alert<2>{0.011}& 0.751 & 0.004\\ \hline
BF(bands=10) & 0.323 & \alert<2>{0.016} & 0.754 & 0.003\\ \hline
BF(bands=5) & 0.468 & \alert<2>{0.113} & 0.768 & 0.003\\ \hline
BF(bands=2) & 0.712 & \alert<2>{0.062} & 0.750 & 0.004

        \end{tabular}
    \end{center}
    \label{tbl:auc_overal}
\end{table}
}
	\item Mahalanobis distance used
\end{itemize}
\end{frame}

\mode<all>{
\usebackgroundtemplate{\includegraphics[height=0.95\paperheight]{img/roc_flow_overal}}
{\footnotesize discriminating http from vpn, httptunnel and woodpecker in \textbf{flow} sample}
\begin{frame}[plain]
\end{frame}
}
\mode<all>{
\usebackgroundtemplate{\includegraphics[height=0.95\paperheight]{img/roc_trace_overal}}
{\footnotesize discriminating http from vpn, httptunnel and woodpecker in \textbf{trace} sample}
\begin{frame}[plain]
\end{frame}
}
\mode<all>{\usebackgroundtemplate{}}
\mode*


\begin{frame}{Model complexity}
\begin{itemize}
	\item Mean and standard deviation of the area under ROC curve. 
    Effect of Gaussian Mixture Model complexity on discriminative performance.
{\small

\begin{table}[h]
    \begin{center}
        \begin{tabular}{c|cc|cc}
        \multirow{2}{*}{\textbf{Method}}  &\multicolumn{2}{ c| }{\emph{trace}} &\multicolumn{2}{ c }{\emph{flow}}\\
         & $\overline{AUC}$ & $s_{AUC}$ & $\overline{AUC}$ & $s_{AUC}$ \\ \hline

\alert<2>{PCA(dim=3),Mahalanobis} & \alert<2>{0.751} & \alert<2>{0.003} & 0.760 & 0.085\\ \hline
\alert<2>{PCA(dim=3),GMM(comp=1)} & \alert<2>{0.751} & \alert<2>{0.003} & 0.760 & 0.085\\ \hline
\alert<3>{PCA(dim=3),GMM(comp=2)} & 0.701 & 0.020 & \alert<3>{0.830} & \alert<3>{0.019}\\ \hline
PCA(dim=3),GMM(comp=5) & 0.532 & 0.014 & 0.821 & 0.031\\ \hline
PCA(dim=3),GMM(comp=10) & 0.590 & 0.021 & 0.827 & 0.019

        \end{tabular}
    \end{center}
  \label{tbl:roc_gmm}
\end{table}
}
	\item PCA transformation with $m=3$ used
\end{itemize}
\end{frame}

\mode<all>{
\usebackgroundtemplate{\includegraphics[height=0.95\paperheight]{img/roc_flow_gmm}}
{\footnotesize discriminating http from vpn, httptunnel and woodpecker in \textbf{flow} sample}
\begin{frame}[plain]
\end{frame}
}
\mode<all>{
\usebackgroundtemplate{\includegraphics[height=0.95\paperheight]{img/roc_trace_gmm}}
{\footnotesize discriminating http from vpn, httptunnel and woodpecker in \textbf{trace} sample}
\begin{frame}[plain]
\end{frame}
}
\mode<all>{\usebackgroundtemplate{}}
\mode*

\begin{frame}{Classes}
\begin{itemize}
	\item Mean and standard deviation of the area under ROC curve.
     Comparison of discriminative performance between clases.
{\small
\begin{table}[h]
    \begin{center}
        \begin{tabular}{c|cc|cc}
        \multirow{2}{*}{\textbf{Method}}  &\multicolumn{2}{ c| }{\emph{flow}} &\multicolumn{2}{ c }{\emph{trace}}\\
         & $\overline{AUC}$ & $s_{AUC}$ & $\overline{AUC}$ & $s_{AUC}$ \\ \hline
vpn & 0.830 & 0.019 & 0.758 & 0.006\\ \hline
woodpecker & 0.796 & 0.107 & 0.816 & 0.009\\ \hline
httptunnel & 0.442 & 0.198 & 0.707 & 0.005\\ \hline
all & 0.814 & 0.210 & 0.755 & 0.005\\ \hline
        \end{tabular}
    \end{center}
    \label{tbl:auc_class}
\end{table}
}
	\item PCA transformation with $m=3$ used
	\item \textbf{trace} sample: Mahalanobis distance used 
	\item \textbf{flow} sample: log-probability under GMM  (2 components) used
\end{itemize}
\end{frame}

\mode<all>{
\usebackgroundtemplate{\includegraphics[height=0.95\paperheight]{img/roc_flow_pca}}
{\footnotesize discriminating http from vpn, httptunnel and woodpecker in \textbf{flow} sample}
\begin{frame}[plain]
\end{frame}
}
\mode<all>{
\usebackgroundtemplate{\includegraphics[height=0.95\paperheight]{img/roc_trace_pca}}
{\footnotesize discriminating http from vpn, httptunnel and woodpecker in \textbf{trace} sample}
\begin{frame}[plain]
\end{frame}
}
\mode<all>{
\usebackgroundtemplate{\includegraphics[height=0.95\paperheight]{img/pca_eigenvalues}}
{\footnotesize semilogtarithmic plot of the eigenvalues using the \emph{flow} and \emph{trace} datasets}
\begin{frame}[plain]
\end{frame}
}
\mode<all>{\usebackgroundtemplate{}}
\mode*


\begin{frame}{References}

{\small
	\bibliographystyle{ieeetr}
	\bibliography{references}
	}
\end{frame}

\end{document}