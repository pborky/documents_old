%%*************************************************************************
%%
%% Optimalization of the Assembly Line Manufacturing Processes
%% 
%% 2012/03/05
%% by Peter Boraros
%% See http://www.pborky.sk/contact for current contact information.
%%
%%*************************************************************************
%%
%% Legal Notice:
%%
%% This code is offered as-is without any warranty either expressed or
%% implied; without even the implied warranty of MERCHANTABILITY or
%% FITNESS FOR A PARTICULAR PURPOSE! 
%% User assumes all risk.
%%
%% This work by Peter Boraros is licensed under a 
%% Creative Commons Attribution-NonCommercial-ShareAlike 3.0 Unported License.
%% http://creativecommons.org/licenses/by-nc-sa/3.0/
%%
%%*************************************************************************

\documentclass[a4paper,journal,twocolumn]{IEEEtran}

\usepackage{cite}
% \usepackage[nocompress]{cite}
\usepackage{ifpdf}

\ifpdf
\usepackage[pdftex]{graphicx}
\graphicspath{{./img/}}
\DeclareGraphicsExtensions{.pdf}
\else
\usepackage[dvips]{graphicx}
\graphicspath{{./img/}}
\DeclareGraphicsExtensions{.eps}
\fi

\usepackage[margin=25mm]{geometry}

\usepackage{graphviz}

\usepackage[cmex10]{amsmath}
\usepackage{amsfonts}
\usepackage{amssymb}
\interdisplaylinepenalty=2500

\usepackage{algorithmic}

\usepackage{array}

\usepackage{mdwmath}
\usepackage{mdwtab}

\usepackage{eqparbox}

\usepackage[hang,small,center,bf]{caption}
% \usepackage[tight,normalsize,sf,SF]{subfigure}
%\usepackage[tight,footnotesize]{subfigure}
\usepackage{subfig}
% \usepackage[caption=false,font=normalsize,labelfont=sf,textfont=sf]{subfig}
% \usepackage[caption=false,font=footnotesize]{subfig}

\usepackage[utf8x]{inputenc}
%\usepackage[czech]{babel}
%\usepackage[T1]{fontenc}


%\usepackage{url}
\usepackage[colorlinks=true,urlcolor=blue]{hyperref}
\usepackage{fixltx2e}
\usepackage{stfloats}
\usepackage{ucs}
\usepackage{multirow}

% correct bad hyphenation here
\hyphenation{op-tical net-works semi-conduc-tor}

\renewcommand{\labelitemi}{$\bullet$}
\renewcommand{\labelitemii}{$\circ$}
\renewcommand{\labelitemiii}{$\ast$}

\setlength{\textheight}{260mm}

\begin{document}

\title{Optimalization of the Manufacturing Processes on the Assembly Line}
\date{March 05, 2012}
\author{Peter~Boráros %
\thanks{{Peter Boráros}, Czech Technical University in Prague,
see~\href{http://www.pborky.sk/contact}{\scriptsize{\texttt{http://www.pborky.sk/contact}}} for a contact infomation}}%

% The paper headers
%\markboth{Peter Boráros, Czech technical university, Faculty of Electrical Engineering, Prague, Czech Republic}{}

%\IEEEcompsoctitleabstractindextext{%
%\begin{abstract}

%\end{abstract}}

\maketitle
\IEEEdisplaynotcompsoctitleabstractindextext
\IEEEpeerreviewmaketitle

\section{The problem}\label{ass}
The task is to design and implement scheduling algorithm for assembly manufacturing line group.
Group is described by directed acyclic graph with one output and several input nodes (see
~\emph{fig.~\ref{fig:graph}}). Input nodes repesent input buffer of the material and output node represents
output buffer of the product. Inner nodes represent assembly machines or control points.
One or more machines or control points are handled by operator. Operator must be qualified for 
position where he works. Operator`s skills can affect the productivity.

Manufacturing line group can produce several models of the product based on setup of the 
machines. Modification to the setup takes defined time and consumes resources of the
technical support group (technicians and engineers).

Given is material availability, presence of the operator and technical staff, the production plan,
the production rate for particular machines, the probability of the defective material in 
particular input buffer and the probability of the defect in process of production for particular
checkpoint.

Output of algorithm is time schedule of the production setups for particular machines as the 
placement of the operator and technical staff (it must be capable to handle with lack of staff).
Algorithm maximizes the production quantity. The constraints are shipment schedule to
the customer and estimate of material availability. Time is discrete and time step is given.


The problem can be categorised as job-shop scheduling problem. The tasks are production orders (instructions for the operators).
The production order (task) depicts an ammount produced model (or duration of the production) and the description of the model
at the particular time span.
Jobs are sets of tasks for particular resource.
The resources are prodution machines and material or subassemblies used by machine.
Availability of the resources is limited by setup or maintence process or by production of the subassemblies
on the upstream assembly line or by shipment of the material from the suppliers.
%Limitations of the availability can be caused by the change of particular task for another taks.

\section{Review}
Most common methodology in job-shop scheduling is material resource planning (MRP). As the MRP is the planning tool it
is not designed for the detailed scheduling. In practice the scheduling is performed by experienced shop-floor
personel with pencil. 
For solving immediate problems, such as sequencing, dispatching rules are often used. 
There have been implemented dispatching rules and heuristics  based on  due dates, criticalicity of the operations, 
processing times and resource utilization. 
Such heuristics and rules were used also in artificial intelligence approaches.\cite{Hoi}

Other methods have been developed based on optimalization methodologies such as \emph{dynamic programming} or 
\emph{branch-and-bound} method.
These methods require at least partial enumeration of possible sequences and thus computational complexity grows exponentially
as the problem size increases. Additionally the schedule is no longer optimal after minor changes in tasks or addition of the 
new tasks or breakdown of the machine.\cite{Hoi}

\emph{Langragian relaxation} method has been developped to solve scheduling problem. 
Method is used to decompose the problem into set of smaller subproblems, 
providing tight lower bound on optimal cost.\cite{Hoi}
The langragian problem can be used to probide bounds in branch-and-bounds algorithm.
This approach lead to increase performance for several problems such as routing problem, scheduling and set covering.\cite{Fis}


\section{Problem solution}
Given a set of $m$ machines and set of $n$ jobs. Each job consist of a sequence of operations each of which needs 
to be processed during an uninterupted time period of a given length on a given resource. 
Each resource can process at most one operation at a time.


\begin{thebibliography}{1}

\bibitem{Hoi} \textsc{D.J. Hoitomt}, \textsc{P.B. Luh}, and \textsc{K.R. Pattipati}, \emph{Practical Approach to Job-Shop Scheduling
Problems}, 1-13, IEEE Trans Rob Autom, Feb, 9(1), 1993.

\bibitem{Fis} \textsc{Marshall L. Fisher}, \emph{The langragian relaxation method for solving linear integer problems}, 1-17, 
Management Science, Vol. 27, No. 1, 1981.

\bibitem{Sed} \textsc{Miloš Šeda}, \emph{Mathematical Models of Flow Shop and Job Shop Scheduling Problem}, 122-127, World Academy of Science, Engineering and Technology 31, 2007.

%\bibitem{} \textsc{}, \emph{}, , .

\end{thebibliography}

\clearpage
\onecolumn
\appendix

\begin{figure}[h]%
  \centering
  \digraph[width=120mm]{MyGraph}{
    rankdir=LR;
	subgraph cluster_0 {
		node [style=filled];
		a0 [shape=square];
		a2 [shape=square];
		a1 [shape=diamond];
		a3 [shape=diamond];
		a0 -> a1 -> a4;
		a2 -> a3 -> a4;
		label = "subassembly";
		color=lightblue;
		style=filled;
	} 
	subgraph cluster_1 {
		node [style=filled];
		b0 [shape=square];
		b1 [shape=diamond];
		b0 -> b1 -> b2 -> b4;
		label = "PCB";
		color=lightblue;
		style=filled;
	}  
	subgraph cluster_2 {
		node [style=filled];
		c0 [shape=square];
		c1 [shape=diamond];
		c0 -> c1 -> c2;
		label = "heatsink";
		color=lightblue;
		style=filled;
	}
	subgraph cluster_3 {
		node [style=filled];
		d0 [shape=square];
		d1 [shape=diamond];
		d2 [shape=diamond];
		d3 [shape=diamond];
		d0 -> d1 -> d4;
		d2 -> d4;
		d3 -> d4;
		label = "assembly";
		color=lightblue;
		style=filled;
	}
	subgraph cluster_4 {
  		node [style=filled];
  		f1 [shape=diamond];
		f1 -> f2;
		label = "packing";
		color=lightblue;
		style=filled;
	}
	b4 -> c2 -> d3;
	a4 -> d2;
	d4 -> f1;
  }
  \caption{Example of production line group. Blue clusters represent particular assembly lines
  and operator position. 
  The input buffers are nodes of square shape and checkpoints are of diamond shape.}
  \label{fig:graph}
\end{figure}

% that's all folks
\end{document}
